\documentclass[11pt,twoside]{article}

\usepackage{cours0}
\usepackage{geometry}
\usepackage{graphicx}
\usepackage{eurosym}

\newcommand{\cbox}[1]{\parbox[t]{1.6cm}{\centering #1}}
\newcommand{\vect}[1]{\boldsymbol{#1}}	
\geometry{hmargin=3cm, vmargin=2.5cm}


%\newcommand{\myauthor}{Laetitia { \sc Chapel}}
%\newcommand{\myemail}{laetitia.chapel@univ-ubs.fr}
%%\newcommand{\documentadress}{http://www.math.jussieu.fr/$\sim$gabriel}
%\newcommand{\mydate}{8 f�vrier 2013}
%\newcommand{\myinstitution}{STID 2}
%\newcommand{\Titre}{ \Large\textbf{Scoring} }


\newcommand{\myyear}{2016}
\newcommand{\myschoolyear}{2019--2020}
\newcommand{\mydate}{Janvier 2020}
\newcommand{\mymodule}{R�gression logistique}
\newcommand{\myinstitution}{DUT STID}
\newcommand{\Titre}{TD1 -- R�gression logistique}


\begin{document}
\thispagestyle{plain}
\mytitle{\Titre}

\newcounter{exo_counter}




\noindent\textbf{Mod�le de r�gression logistique}
\par 
\bigskip
\begin{enumerate}
\item Montrer que les formulations 
$$\pi(x) = \dfrac{\exp(\beta_0 + \beta_1\vect{x_1}+ \cdots + \beta_p\vect{x_p})}{1+\exp(\beta_0 + \beta_1\vect{x_1}+ \cdots + \beta_p\vect{x_p})} $$
et 
$$\text{logit}(\pi(x)) = \log\left(\dfrac{\pi(x)}{1-\pi(x)} \right)= \beta_0 + \beta_1\vect{x_1}+ \cdots + \beta_p\vect{x_p} $$
sont �quivalentes.

\end{enumerate}

\end{document}